\documentclass{article}
\usepackage{hyperref}
\usepackage{pgfgantt}
\begin{document}

\title{Advanced Computer Graphics}
\author{Oliver Eisenberg}
\maketitle
\pagebreak
\tableofcontents
\pagebreak
\twocolumn
\section{Labs}
\subsection{Lab 1 - Rasterising Lines}

\subsection{Lab 2 - Reading Models}

\subsection{Lab 3 - Simple Raytracing}
\subsubsection{Raycasting }
\subsubsection{Triangle intersection}
To compute triangle intersections the Möller–Trumbore algorithm was used. This was used instead of the method on the \textit{slides} anticipating the requirements for the baracentric coordinates to complete Gouraud shading further in the coursework.

\subsection{Lab 4 - Basic Lighting and Shadows}
\subsubsection{Spotlights}

\subsubsection{Pointlights}
The slides refer to two methods to create pointlights, with and without an associated direction. As spot lights are directional, pointlights with a constant intensity were implemented. This allows a pointlight to be placed between objects to cast shadow outwa
\subsubsection{Shadows}

\section{Optimisations}

\section{Advanced Features}



\end{document}
